\section{BEADS}
\label{BEADS}
\begin{ipifield}{}%
{Describes the bead configurations in a path integral simulation.}%
{}%
{\ipiitem{natoms}%
{The number of atoms.}%
{default:  0 ; data type: integer; }%
\ipiitem{nbeads}%
{The number of beads.}%
{default:  0 ; data type: integer; }%
}
\begin{ipifield}{q}%
{The positions of the beads. In an array of size [nbeads, 3*natoms].}%
{dimension: length; default:  [ ] ; data type: float; }%
{\ipiitem{units}%
{The units the input data is given in.}%
{default: `automatic'; data type: string; }%
\ipiitem{shape}%
{The shape of the array.}%
{default:  (0,) ; data type: tuple; }%
\ipiitem{mode}%
{If 'mode' is 'manual', then the array is read from the content of 'cell' takes a 9-elements vector containing the cell matrix (row-major). If 'mode' is 'abcABC', then 'cell' takes an array of 6 floats, the first three being the length of the sides of the system parallelopiped, and the last three being the angles (in degrees) between those sides. Angle A corresponds to the angle between sides b and c, and so on for B and C. If mode is 'abc', then this is the same as for 'abcABC', but the cell is assumed to be orthorhombic. 'pdb' and 'chk' read the cell from a PDB or a checkpoint file, respectively.}%
{default: `manual'; data type: string; options: `manual', `file'; }%
}
\end{ipifield}
\begin{ipifield}{p}%
{The momenta of the beads. In an array of size [nbeads, 3*natoms].}%
{dimension: momentum; default:  [ ] ; data type: float; }%
{\ipiitem{units}%
{The units the input data is given in.}%
{default: `automatic'; data type: string; }%
\ipiitem{shape}%
{The shape of the array.}%
{default:  (0,) ; data type: tuple; }%
\ipiitem{mode}%
{If 'mode' is 'manual', then the array is read from the content of 'cell' takes a 9-elements vector containing the cell matrix (row-major). If 'mode' is 'abcABC', then 'cell' takes an array of 6 floats, the first three being the length of the sides of the system parallelopiped, and the last three being the angles (in degrees) between those sides. Angle A corresponds to the angle between sides b and c, and so on for B and C. If mode is 'abc', then this is the same as for 'abcABC', but the cell is assumed to be orthorhombic. 'pdb' and 'chk' read the cell from a PDB or a checkpoint file, respectively.}%
{default: `manual'; data type: string; options: `manual', `file'; }%
}
\end{ipifield}
\begin{ipifield}{m}%
{The masses of the atoms, in the format [m1, m2, \ldots  ].}%
{dimension: mass; default:  [ ] ; data type: float; }%
{\ipiitem{units}%
{The units the input data is given in.}%
{default: `automatic'; data type: string; }%
\ipiitem{shape}%
{The shape of the array.}%
{default:  (0,) ; data type: tuple; }%
\ipiitem{mode}%
{If 'mode' is 'manual', then the array is read from the content of 'cell' takes a 9-elements vector containing the cell matrix (row-major). If 'mode' is 'abcABC', then 'cell' takes an array of 6 floats, the first three being the length of the sides of the system parallelopiped, and the last three being the angles (in degrees) between those sides. Angle A corresponds to the angle between sides b and c, and so on for B and C. If mode is 'abc', then this is the same as for 'abcABC', but the cell is assumed to be orthorhombic. 'pdb' and 'chk' read the cell from a PDB or a checkpoint file, respectively.}%
{default: `manual'; data type: string; options: `manual', `file'; }%
}
\end{ipifield}
\begin{ipifield}{names}%
{The names of the atoms, in the format [name1, name2, \ldots  ].}%
{default:  [ ] ; data type: string; }%
{\ipiitem{shape}%
{The shape of the array.}%
{default:  (0,) ; data type: tuple; }%
\ipiitem{mode}%
{If 'mode' is 'manual', then the array is read from the content of 'cell' takes a 9-elements vector containing the cell matrix (row-major). If 'mode' is 'abcABC', then 'cell' takes an array of 6 floats, the first three being the length of the sides of the system parallelopiped, and the last three being the angles (in degrees) between those sides. Angle A corresponds to the angle between sides b and c, and so on for B and C. If mode is 'abc', then this is the same as for 'abcABC', but the cell is assumed to be orthorhombic. 'pdb' and 'chk' read the cell from a PDB or a checkpoint file, respectively.}%
{default: `manual'; data type: string; options: `manual', `file'; }%
}
\end{ipifield}
\end{ipifield}
