\section{FFSOCKET}
\label{FFSOCKET}
\begin{ipifield}{}%
{Deals with the assigning of force calculation jobs to different driver codes, and collecting the data, using a socket for the data communication.}%
{}%
{\ipiitem{pbc}%
{Applies periodic boundary conditions to the atoms coordinates before passing them on to the driver code.}%
{default:  True ; data type: boolean; }%
\ipiitem{mode}%
{Specifies whether the driver interface will listen onto a internet socket [inet] or onto a unix socket [unix].}%
{default: `inet'; data type: string; options: `unix', `inet'; }%
\ipiitem{name}%
{Mandatory. The name by which the forcefield will be identified in the System forces section.}%
{data type: string; }%
\ipiitem{matching}%
{Specifies whether requests should be dispatched to any client, or automatically matched to the same client when possible [auto].}%
{default: `auto'; data type: string; options: `auto', `any'; }%
}
\begin{ipifield}{latency}%
{The number of seconds the polling thread will wait between exhamining the list of requests.}%
{default:  0.01 ; data type: float; }%
{}
\end{ipifield}
\begin{ipifield}{timeout}%
{This gives the number of seconds before assuming a calculation has died. If 0 there is no timeout.}%
{default:  0.0 ; data type: float; }%
{}
\end{ipifield}
\begin{ipifield}{parameters}%
{The parameters of the force field}%
{default:  \{ \} ; data type: dictionary; }%
{}
\end{ipifield}
\begin{ipifield}{address}%
{This gives the server address that the socket will run on.}%
{default: `localhost'; data type: string; }%
{}
\end{ipifield}
\begin{ipifield}{activelist}%
{List with indexes of the atoms that this socket is taking care of.    Default: all (corresponding to -1)}%
{default: 
      [-1]; data type: integer; }%
{\ipiitem{shape}%
{The shape of the array.}%
{default:  (1,) ; data type: tuple; }%
\ipiitem{mode}%
{If 'mode' is 'manual', then the array is read from the content of 'cell' takes a 9-elements vector containing the cell matrix (row-major). If 'mode' is 'abcABC', then 'cell' takes an array of 6 floats, the first three being the length of the sides of the system parallelopiped, and the last three being the angles (in degrees) between those sides. Angle A corresponds to the angle between sides b and c, and so on for B and C. If mode is 'abc', then this is the same as for 'abcABC', but the cell is assumed to be orthorhombic. 'pdb' and 'chk' read the cell from a PDB or a checkpoint file, respectively.}%
{default: `manual'; data type: string; options: `manual', `file'; }%
}
\end{ipifield}
\begin{ipifield}{slots}%
{This gives the number of client codes that can queue at any one time.}%
{default:  4 ; data type: integer; }%
{}
\end{ipifield}
\begin{ipifield}{port}%
{This gives the port number that defines the socket.}%
{default:  65535 ; data type: integer; }%
{}
\end{ipifield}
\end{ipifield}
