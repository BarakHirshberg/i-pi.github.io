\section{NMFREQUENCIES}
\label{NMFREQUENCIES}
\begin{ipifield}{}%
{Provides a compact way of specifying the ring polymer frequencies}%
{dimension: frequency; data type: float; }%
{\ipiitem{units}%
{The units the input data is given in.}%
{default: `automatic'; data type: string; }%
\ipiitem{shape}%
{The shape of the array.}%
{default:  (0,) ; data type: tuple; }%
\ipiitem{style}%
{Specifies the technique to be used to calculate the dynamical masses.
                                                'rpmd' simply assigns the bead masses the physical mass.
                                                'manual' sets all the normal mode frequencies except the centroid normal mode manually.
                                                'pa-cmd' takes an argument giving the frequency to set all the non-centroid normal modes to.
                                                'wmax-cmd' is similar to 'pa-cmd', except instead of taking one argument it takes two
                                                      ([wmax,wtarget]). The lowest-lying normal mode will be set to wtarget for a
                                                      free particle, and all the normal modes will coincide at frequency wmax. }%
{default: `rpmd'; data type: string; options: `pa-cmd', `wmax-cmd', `manual', `rpmd'; }%
\ipiitem{mode}%
{If 'mode' is 'manual', then the array is read from the content of 'cell' takes a 9-elements vector containing the cell matrix (row-major). If 'mode' is 'abcABC', then 'cell' takes an array of 6 floats, the first three being the length of the sides of the system parallelopiped, and the last three being the angles (in degrees) between those sides. Angle A corresponds to the angle between sides b and c, and so on for B and C. If mode is 'abc', then this is the same as for 'abcABC', but the cell is assumed to be orthorhombic. 'pdb' and 'chk' read the cell from a PDB or a checkpoint file, respectively.}%
{default: `manual'; data type: string; options: `manual', `file'; }%
}
\end{ipifield}
