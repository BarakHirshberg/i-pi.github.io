\section{MOTION}
\label{MOTION}
\begin{ipifield}{}%
{Allow chosing the type of calculation to be performed. Holds all the information that is calculation specific, such as geometry optimization parameters, etc.}%
{}%
{\ipiitem{mode}%
{How atoms should be moved at each step in the simulatio. 'replay' means that a simulation is restarted from a previous simulation.}%
{data type: string; options: `vibrations', `minimize', `replay', `neb', `dynamics', `alchemy', `instanton', `dummy', `multi'; }%
}
\begin{ipifield}{\hyperref[GEOP]{optimizer}}%
{Option for geometry optimization}%
{}%
{\ipiitem{mode}%
{The geometry optimization algorithm to be used}%
{default: `lbfgs'; data type: string; options: `sd', `cg', `bfgs', `bfgstrm', `lbfgs'; }%
}
\end{ipifield}
\begin{ipifield}{fixatoms}%
{Indices of the atmoms that should be held fixed.}%
{default:  [ ] ; data type: integer; }%
{\ipiitem{shape}%
{The shape of the array.}%
{default:  (0,) ; data type: tuple; }%
\ipiitem{mode}%
{If 'mode' is 'manual', then the array is read from the content of 'cell' takes a 9-elements vector containing the cell matrix (row-major). If 'mode' is 'abcABC', then 'cell' takes an array of 6 floats, the first three being the length of the sides of the system parallelopiped, and the last three being the angles (in degrees) between those sides. Angle A corresponds to the angle between sides b and c, and so on for B and C. If mode is 'abc', then this is the same as for 'abcABC', but the cell is assumed to be orthorhombic. 'pdb' and 'chk' read the cell from a PDB or a checkpoint file, respectively.}%
{default: `manual'; data type: string; options: `manual', `file'; }%
}
\end{ipifield}
\begin{ipifield}{\hyperref[ALCHEMY]{alchemy}}%
{Option for alchemical exchanges}%
{}%
{\ipiitem{mode}%
{ }%
{default: `dummy'; data type: string; options: `dummy'; }%
}
\end{ipifield}
\begin{ipifield}{\hyperref[PHONONS]{vibrations}}%
{Option for phonon computation}%
{}%
{\ipiitem{mode}%
{The algorithm to be used: finite differences (fd), normal modes finite differences (nmfd), and energy-scaled normal mode finite differences (enmfd).}%
{default: `fd'; data type: string; options: `fd', `nmfd', `enmfd'; }%
}
\end{ipifield}
\begin{ipifield}{\hyperref[INSTANTON]{instanton}}%
{Option for Instanton optimization}%
{}%
{\ipiitem{mode}%
{Use the full or half of the ring polymer during the optimization}%
{default: `rate'; data type: string; options: `rate', `splitting'; }%
}
\end{ipifield}
\begin{ipifield}{\hyperref[INITFILE]{file}}%
{This describes the location to read a trajectory file from.}%
{default: `'; data type: string; }%
{\ipiitem{bead}%
{The index of the bead for which the value will be set. If a negative value is specified, then all beads are assumed.}%
{default:  -1 ; data type: integer; }%
\ipiitem{cell\_units}%
{The units for the cell dimensions.}%
{default: `'; data type: string; }%
\ipiitem{mode}%
{The input data format. 'xyz' and 'pdb' stand for xyz and pdb input files respectively. 'chk' stands for initialization from a checkpoint file.}%
{default: `xyz'; data type: string; options: `xyz', `pdb', `chk'; }%
}
\end{ipifield}
\begin{ipifield}{\hyperref[NEB]{neb\_optimizer}}%
{Option for geometry optimization}%
{}%
{\ipiitem{mode}%
{The geometry optimization algorithm to be used}%
{default: `lbfgs'; data type: string; options: `sd', `cg', `bfgs', `lbfgs'; }%
}
\end{ipifield}
\begin{ipifield}{\hyperref[DYNAMICS]{dynamics}}%
{Option for (path integral) molecular dynamics}%
{}%
{\ipiitem{mode}%
{The ensemble that will be sampled during the simulation. }%
{default: `nve'; data type: string; options: `nve', `nvt', `npt', `nst', `sc', `scnpt'; }%
\ipiitem{splitting}%
{The Louiville splitting used for sampling the target ensemble. }%
{default: `obabo'; data type: string; options: `obabo', `baoab'; }%
}
\end{ipifield}
\begin{ipifield}{fixcom}%
{This describes whether the centre of mass of the particles is fixed.}%
{default:  True ; data type: boolean; }%
{}
\end{ipifield}
\begin{ipifield}{\hyperref[MOTION]{motion}}%
{A motion class that can be included as a member of a 'multi' integrator.}%
{}%
{\ipiitem{mode}%
{How atoms should be moved at each step in the simulatio. 'replay' means that a simulation is restarted from a previous simulation.}%
{data type: string; options: `vibrations', `minimize', `replay', `neb', `dynamics', `alchemy', `instanton', `dummy', `multi'; }%
}
\end{ipifield}
\end{ipifield}
