\section{FORCES}
\label{FORCES}
\begin{ipifield}{}%
{Deals with creating all the necessary forcefield objects.}%
{}%
{}
\begin{ipifield}{\hyperref[FORCECOMPONENT]{force}}%
{The class that deals with how each forcefield contributes to the overall potential, force and virial calculation.}%
{}%
{\ipiitem{forcefield}%
{Mandatory. The name of the forcefield this force is referring to.}%
{default: `'; data type: string; }%
\ipiitem{fd\_epsilon}%
{The finite displacement to be used for calculaing the Suzuki-Chin contribution of the force. If the value is negative, a centered finite-difference scheme will be used. [in bohr]}%
{default:  -0.001 ; data type: float; }%
\ipiitem{name}%
{An optional name to refer to this force component.}%
{default: `'; data type: string; }%
\ipiitem{weight}%
{A scaling factor for this forcefield, to be applied before adding the force calculated by this forcefield to the total force.}%
{default:  1.0 ; data type: float; }%
\ipiitem{nbeads}%
{If the forcefield is to be evaluated on a contracted ring polymer, this gives the number of beads that are used. If not specified, the forcefield will be evaluated on the full ring polymer.}%
{default:  0 ; data type: integer; }%
}
\end{ipifield}
\end{ipifield}
