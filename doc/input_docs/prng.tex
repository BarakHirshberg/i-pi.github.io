\section{PRNG}
\label{PRNG}
\begin{ipifield}{}%
{Deals with the pseudo-random number generator.}%
{}%
{}
\begin{ipifield}{set\_pos}%
{Gives the position in the state array that the random number generator is reading from.}%
{default:  0 ; data type: integer; }%
{}
\end{ipifield}
\begin{ipifield}{state}%
{Gives the state vector for the random number generator. Avoid directly modifying this unless you are very familiar with the inner workings of the algorithm used.}%
{default:  [ ] ; data type: integer; }%
{\ipiitem{shape}%
{The shape of the array.}%
{default:  (0,) ; data type: tuple; }%
\ipiitem{mode}%
{If 'mode' is 'manual', then the array is read from the content of 'cell' takes a 9-elements vector containing the cell matrix (row-major). If 'mode' is 'abcABC', then 'cell' takes an array of 6 floats, the first three being the length of the sides of the system parallelopiped, and the last three being the angles (in degrees) between those sides. Angle A corresponds to the angle between sides b and c, and so on for B and C. If mode is 'abc', then this is the same as for 'abcABC', but the cell is assumed to be orthorhombic. 'pdb' and 'chk' read the cell from a PDB or a checkpoint file, respectively.}%
{default: `manual'; data type: string; options: `manual', `file'; }%
}
\end{ipifield}
\begin{ipifield}{has\_gauss}%
{Determines whether there is a stored gaussian number or not. A value of 0 means there is none stored.}%
{default:  0 ; data type: integer; }%
{}
\end{ipifield}
\begin{ipifield}{seed}%
{This is the seed number used to generate the initial state of the random number generator.}%
{default:  123456 ; data type: integer; }%
{}
\end{ipifield}
\begin{ipifield}{gauss}%
{The stored Gaussian number.}%
{default:  0.0 ; data type: float; }%
{}
\end{ipifield}
\end{ipifield}
