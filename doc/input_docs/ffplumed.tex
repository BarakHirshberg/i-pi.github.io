\section{FFPLUMED}
\label{FFPLUMED}
\begin{ipifield}{}%
{ Direct PLUMED interface }%
{}%
{\ipiitem{name}%
{Mandatory. The name by which the forcefield will be identified in the System forces section.}%
{data type: string; }%
\ipiitem{pbc}%
{Applies periodic boundary conditions to the atoms coordinates before passing them on to the driver code.}%
{default:  True ; data type: boolean; }%
}
\begin{ipifield}{latency}%
{The number of seconds the polling thread will wait between exhamining the list of requests.}%
{default:  0.01 ; data type: float; }%
{}
\end{ipifield}
\begin{ipifield}{parameters}%
{The parameters of the force field}%
{default:  \{ \} ; data type: dictionary; }%
{}
\end{ipifield}
\begin{ipifield}{plumeddat}%
{The PLUMED input file}%
{default: `plumed.dat'; data type: string; }%
{}
\end{ipifield}
\begin{ipifield}{activelist}%
{List with indexes of the atoms that this socket is taking care of.    Default: all (corresponding to -1)}%
{default: 
      [-1]; data type: integer; }%
{\ipiitem{shape}%
{The shape of the array.}%
{default:  (1,) ; data type: tuple; }%
\ipiitem{mode}%
{If 'mode' is 'manual', then the array is read from the content of 'cell' takes a 9-elements vector containing the cell matrix (row-major). If 'mode' is 'abcABC', then 'cell' takes an array of 6 floats, the first three being the length of the sides of the system parallelopiped, and the last three being the angles (in degrees) between those sides. Angle A corresponds to the angle between sides b and c, and so on for B and C. If mode is 'abc', then this is the same as for 'abcABC', but the cell is assumed to be orthorhombic. 'pdb' and 'chk' read the cell from a PDB or a checkpoint file, respectively.}%
{default: `manual'; data type: string; options: `manual', `file'; }%
}
\end{ipifield}
\begin{ipifield}{plumedstep}%
{The current step counter for PLUMED calls}%
{default:  0 ; data type: integer; }%
{}
\end{ipifield}
\begin{ipifield}{\hyperref[INITFILE]{init\_file}}%
{This describes the location to read the reference structure file from.}%
{default: `'; data type: string; }%
{\ipiitem{bead}%
{The index of the bead for which the value will be set. If a negative value is specified, then all beads are assumed.}%
{default:  -1 ; data type: integer; }%
\ipiitem{cell\_units}%
{The units for the cell dimensions.}%
{default: `'; data type: string; }%
\ipiitem{mode}%
{The input data format. 'xyz' and 'pdb' stand for xyz and pdb input files respectively. 'chk' stands for initialization from a checkpoint file.}%
{default: `xyz'; data type: string; options: `xyz', `pdb', `chk'; }%
}
\end{ipifield}
\begin{ipifield}{precision}%
{The precision PLUMED was compiled with}%
{default:  8 ; data type: integer; }%
{}
\end{ipifield}
\end{ipifield}
