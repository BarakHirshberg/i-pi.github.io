\section{INITIALIZER}
\label{INITIALIZER}
\begin{ipifield}{}%
{Specifies the number of beads, and how the system should be initialized.}%
{}%
{\ipiitem{nbeads}%
{The number of beads. Will override any provision from inside the initializer. A ring polymer contraction scheme is used to scale down the number of beads if required. If instead the number of beads is scaled up, higher normal modes will be initialized to zero.}%
{data type: integer; }%
}
\begin{ipifield}{\hyperref[INITCELL]{cell}}%
{Initializes the configuration of the cell. Will take a 'units' attribute of dimension 'length'}%
{data type: string; }%
{\ipiitem{mode}%
{This decides whether the system box is created from a cell parameter matrix, or from the side lengths and angles between them. If 'mode' is 'manual', then 'cell' takes a 9-elements vector containing the cell matrix (row-major). If 'mode' is 'abcABC', then 'cell' takes an array of 6 floats, the first three being the length of the sides of the system parallelopiped, and the last three being the angles (in degrees) between those sides. Angle A corresponds to the angle between sides b and c, and so on for B and C. If mode is 'abc', then this is the same as for 'abcABC', but the cell is assumed to be orthorhombic. 'pdb' and 'chk' read the cell from a PDB or a checkpoint file, respectively.}%
{default: `manual'; data type: string; options: `manual', `pdb', `chk', `abc', `abcABC'; }%
}
\end{ipifield}
\begin{ipifield}{\hyperref[INITLABELS]{labels}}%
{Initializes atomic labels}%
{data type: string; }%
{\ipiitem{index}%
{The index of the atom for which the value will be set. If a negative value is specified, then all atoms are assumed.}%
{default:  -1 ; data type: integer; }%
\ipiitem{bead}%
{The index of the bead for which the value will be set. If a negative value is specified, then all beads are assumed.}%
{default:  -1 ; data type: integer; }%
\ipiitem{mode}%
{The input data format. 'xyz' and 'pdb' stand for xyz and pdb input files respectively. 'chk' stands for initialization from a checkpoint file. 'manual' means that the value to initialize from is giving explicitly as a vector.}%
{default: `chk'; data type: string; options: `manual', `xyz', `pdb', `chk'; }%
}
\end{ipifield}
\begin{ipifield}{\hyperref[INITFILE]{file}}%
{Initializes everything possible for the given mode. Will take a 'units' attribute of dimension 'length'. The unit conversion will only be applied to the positions and cell parameters. The 'units' attribute is deprecated. Append a 'quantity{units}' to the comment line of the xyz or to the 'TITLE' tag of a pdb.}%
{data type: string; }%
{\ipiitem{bead}%
{The index of the bead for which the value will be set. If a negative value is specified, then all beads are assumed.}%
{default:  -1 ; data type: integer; }%
\ipiitem{cell\_units}%
{The units for the cell dimensions.}%
{default: `automatic'; data type: string; }%
\ipiitem{mode}%
{The input data format. 'xyz' and 'pdb' stand for xyz and pdb input files respectively. 'chk' stands for initialization from a checkpoint file.}%
{default: `chk'; data type: string; options: `xyz', `pdb', `chk'; }%
}
\end{ipifield}
\begin{ipifield}{\hyperref[INITPOSITIONS]{positions}}%
{Initializes atomic positions. Will take a 'units' attribute of dimension 'length'}%
{data type: string; }%
{\ipiitem{index}%
{The index of the atom for which the value will be set. If a negative value is specified, then all atoms are assumed.}%
{default:  -1 ; data type: integer; }%
\ipiitem{bead}%
{The index of the bead for which the value will be set. If a negative value is specified, then all beads are assumed.}%
{default:  -1 ; data type: integer; }%
\ipiitem{mode}%
{The input data format. 'xyz' and 'pdb' stand for xyz and pdb input files respectively. 'chk' stands for initialization from a checkpoint file. 'manual' means that the value to initialize from is giving explicitly as a vector.}%
{default: `chk'; data type: string; options: `manual', `xyz', `pdb', `chk'; }%
}
\end{ipifield}
\begin{ipifield}{\hyperref[INITMOMENTA]{momenta}}%
{Initializes atomic momenta. Will take a 'units' attribute of dimension 'momentum'}%
{data type: string; }%
{\ipiitem{index}%
{The index of the atom for which the value will be set. If a negative value is specified, then all atoms are assumed.}%
{default:  -1 ; data type: integer; }%
\ipiitem{bead}%
{The index of the bead for which the value will be set. If a negative value is specified, then all beads are assumed.}%
{default:  -1 ; data type: integer; }%
\ipiitem{mode}%
{The input data format. 'xyz' and 'pdb' stand for xyz and pdb input files respectively. 'chk' stands for initialization from a checkpoint file. 'manual' means that the value to initialize from is giving explicitly as a vector. 'thermal' means that the data is to be generated from a Maxwell-Boltzmann distribution at the given temperature.}%
{default: `chk'; data type: string; options: `manual', `xyz', `pdb', `chk', `thermal'; }%
}
\end{ipifield}
\begin{ipifield}{\hyperref[INITVELOCITIES]{velocities}}%
{Initializes atomic velocities. Will take a 'units' attribute of dimension 'velocity'}%
{data type: string; }%
{\ipiitem{index}%
{The index of the atom for which the value will be set. If a negative value is specified, then all atoms are assumed.}%
{default:  -1 ; data type: integer; }%
\ipiitem{bead}%
{The index of the bead for which the value will be set. If a negative value is specified, then all beads are assumed.}%
{default:  -1 ; data type: integer; }%
\ipiitem{mode}%
{The input data format. 'xyz' and 'pdb' stand for xyz and pdb input files respectively. 'chk' stands for initialization from a checkpoint file. 'manual' means that the value to initialize from is giving explicitly as a vector. 'thermal' means that the data is to be generated from a Maxwell-Boltzmann distribution at the given temperature.}%
{default: `chk'; data type: string; options: `manual', `xyz', `pdb', `chk', `thermal'; }%
}
\end{ipifield}
\begin{ipifield}{\hyperref[INITMASSES]{masses}}%
{Initializes atomic masses. Will take a 'units' attribute of dimension 'mass'}%
{data type: string; }%
{\ipiitem{index}%
{The index of the atom for which the value will be set. If a negative value is specified, then all atoms are assumed.}%
{default:  -1 ; data type: integer; }%
\ipiitem{bead}%
{The index of the bead for which the value will be set. If a negative value is specified, then all beads are assumed.}%
{default:  -1 ; data type: integer; }%
\ipiitem{mode}%
{The input data format. 'xyz' and 'pdb' stand for xyz and pdb input files respectively. 'chk' stands for initialization from a checkpoint file. 'manual' means that the value to initialize from is giving explicitly as a vector.}%
{default: `chk'; data type: string; options: `manual', `xyz', `pdb', `chk'; }%
}
\end{ipifield}
\begin{ipifield}{\hyperref[INITTHERMO]{gle}}%
{Initializes the additional momenta in a GLE thermostat.}%
{data type: string; }%
{\ipiitem{mode}%
{'chk' stands for initialization from a checkpoint file. 'manual' means that the value to initialize from is giving explicitly as a vector.}%
{default: `manual'; data type: string; options: `chk', `manual'; }%
}
\end{ipifield}
\end{ipifield}
